
%%%%%%%%%%%%%%%
% Define a switch for submission mode
%%%%%%%%%%%%%%%

\newif\ifsubmode
\submodefalse

%%%%%%%%%%%%%%%
% Define a switch to print the figures when in submission mode
%%%%%%%%%%%%%%%

\newif\ifprintfig
\printfigtrue

%%%%%%%%%%%%%%%
% Define a switch to emulate ApJ style
%%%%%%%%%%%%%%%

\newif\ifemulate
\emulatetrue

%%%%%%%%%%%%%%%
% Preamble
%%%%%%%%%%%%%%%

\ifsubmode
  \documentclass[12pt,preprint]{aastex}
  \received{}
  \accepted{}
  \journalid{}{}
  \articleid{}{}
\else
   \documentclass{emulateapj}
\fi
\usepackage{amssymb,amsmath,mathrsfs,hyperref,datetime}

\newcommand{\datavector}[1]{\boldsymbol{#1}}
\newcommand{\flux}{\datavector{F}}
\newcommand{\uncertainty}{\datavector{\sigma_F}}
\newcommand{\hyperpars}{\datavector{\alpha}}
\newcommand{\setofall}[1]{\left\{{#1}\right\}}
\newcommand{\allflux}{\setofall{\flux_i}}
\newcommand{\dd}{\mathrm{d}}
\newcommand\rf[1]{{\bf [RF: #1]}}
\newcommand\bw[1]{{\bf [BW: #1]}}

\def\urltilda{\kern -.15em\lower .7ex\hbox{\~{}}\kern .04em}

\pdfoutput=1

\begin{document}

\title{Joint photometric modeling for improved measurements in imaging surveys.}
\author{Ross~Fadely\altaffilmark{1},
        David~W.~Hogg\altaffilmark{1,2},
        \& Beth~Willman\altaffilmark{3}}
\altaffiltext{1}{Center~for~Cosmology~and~Particle~Physics, Department~of~Physics, New~York~University, 4~Washington~Place, New~York, NY 10003, USA}
\altaffiltext{2}{Max-Planck-Institut f\"ur Astronomie, K\"onigstuhl 17, 69117 Heidelberg, Germany}
\altaffiltext{3}{Haverford College, Department of Physics and Astronomy, 370 Lancaster Ave., Haverford, PA, 19041}


%%%%%%%%%%%%%%%%%%%%%%%%%%%%%%%%%%%%%%%%%%%
%
% 					Abstract
%
%%%%%%%%%%%%%%%%%%%%%%%%%%%%%%%%%%%%%%%%%%%
\begin{abstract}
Large scale astronomical imaging surveys have fundamentally changed our 
understanding of the cosmos, and promise to continue to do so with 
forthcoming surveys like the Large Synoptic Survey Telescope which will 
provide greater volumes of data at increasingly fainter detection limits.  
In order to extract the most science out of such costly missions, it is 
important to continually explore ways to improve the way photometric 
measurements are made.  Historically, photometric measurements are made 
on a per object basis under our understanding of the characteristics of 
the telescope, instruments, and astronomical sources.  We argue that it 
is possible to improve the quality of such photometric measurements by 
building models which incorporate the shared similarity of sources.  As 
a demonstration, we use Extreme Deconvolution (XD) to model the joint 
distribution of photometric quantities measured by SDSS in Stripe 82.  
Using single epoch data, we show that our joint XD model makes predictions 
that more closely resemble the coadded Stripe 82 data.  Looking at the 
difference between the psf and model magnitudes in the single epoch and 
coadded data, the average error is XX at $r\sim21$ versus an average of 
XX for our XD posteriors.  In the case of the CMDs of globular clusters, 
we show that the XD posteriors lessen the presence of misclassified 
galaxies and present a narrower stellar locus.  We suggest a number of 
improvements that can be made to provide improved photometric inferences.
\end{abstract}

%%%%%%%%%%%%%%%%%%%%%%%%%%%%%%%%%%%%%%%%%%%
%
% 					Section - Introduction
%
%%%%%%%%%%%%%%%%%%%%%%%%%%%%%%%%%%%%%%%%%%%
\section{Introduction}



\end{document}
